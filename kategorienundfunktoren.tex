\section{Klassen und Mengen}
Um sinnvoll über Mathemaik im Allgemeinen und Kategorien im Speziellen reden zu können, müssen wir zunächst einige Grundbegriffe bezüglich Mengen und Klassen einführen.

Alle mathematischen Objekte, die wir betrachten sind so genannte \Def{Klassen}. Zwischen zwei Klassen $x$ und $X$ kann die \Def{Elementrelation} $\in$ gelten, wir schreiben dann $x\in X$ und nennen $x$ ein \Def{Element} von $X$. Ist eine Klasse $X$ ein Element einer anderen Klasse, so nennen wir $X$ eine \Def{Menge}. Ist eine Klasse keine Menge, so heißt sie eine \Def{echte Klasse}.

Durch Axiome wollen wir in diesem Kapitel festlegen, wie Klassen und Mengen manipuliert werden können. Wir nennen eine Klasse $X$ \Def{Teilklasse} einer Klasse $Y$, wenn aus $x\in X$ stets $x\in Y$ folgt. Wir schreiben dann $X\subseteq Y$.

\begin{axiom}
Zwei Klassen $X$ und $Y$ sind genau dann gleich, wenn $X\subseteq Y$ und $Y\subseteq X$.
\end{axiom}

Anschaulich besagt dieses Axiom, das \Def{Extensionalitätsaxiom}, dass zwei Klassen genau dann gleich sind, wenn sie dieselben Elemente enthalten.

\begin{axiom}
Jede Teilklasse einer Menge ist selbst eine Menge.
\end{axiom}

Ist $Y$ eine Menge und $X\subseteq Y$ eine Teilklasse von $Y$, so heißt $X$ aufgrund obigen Axioms auch eine \Def{Teilmenge} von $Y$. Obiges Axiom nennen wir das \Def{Teilmengenaxiom}.

\begin{axiom}
Für jede Menge $x$ sei eine Aussage $E(x)$ gegeben, die entweder wahr oder falsch sein kann. Dann gibt es eine Klasse
\[
\{x\mid E(x)\}\,,
\]
welche genau die Mengen $x$ enthält, für welche $E(x)$ wahr ist.
\end{axiom}

Dieses Axiom, das \Def{Aussonderungsaxiom} zusammen mit dem Teilmengenaxiom garantiert, dass, falls $X$ eine Menge ist, auch die Klasse
\[
\{x\mid x\in X\text{ und }E(x)\}
\]
eine Menge ist. Für diese Menge wählen wir die einfachere Notation $\{x\in X\mid E(x)\}$.

\begin{lemma}
Die Klasse $R=\{x\mid x\notin x\}$ ist eine echte Klasse.
\end{lemma}
\begin{proof}
Wäre $R$ eine Menge, so würde aus $R\in R$ folgen, dass $R$ die Eigenschaft $E(R)=R\notin R$ besitzt. Andererseits, wenn $R$ die Eigenschaft $R\notin R$ besitzt, muss $R$ in der Klasse $\{x\mid x\notin x\}$ liegen, also $R\in R$; Widerspruch.
\end{proof}

Nach dem Teilmengenaxiom muss dann auch $U=\{x\mid x=x\}$ eine echte Klasse sein, da sonst auch $R\subseteq U$ eine Menge bilden würde. Wir nennen $U$ die \Def{Allklasse} oder die Klasse aller Mengen.

\begin{axiom}
Es existiert eine Menge.
\end{axiom}

Dieses \Def{Existenzaxiom} sichert, dass wir eine Menge $X$ wählen können. Nach dem Teilmengenaxiom ist dann auch
\[\emptyset=\{x\mid x\not= x\}\] eine Menge, da $\emptyset$ offensichtlich Teilklasse jeder Klasse und insbesondere Teilmenge von $X$ ist. Wir nennen $\emptyset$ die \Def{leere Menge}. Für jede Klasse $x$ gilt $x\notin\emptyset$, da sonst $x\not= x$ gelten würde. Eine Klasse, welche ein Element enthält; also von $\emptyset$ verschieden ist, heißt \Def{nichtleer}.

\begin{axiom}
Für jede Menge $X$ ist auch
\[
\pot X=\{Y\mid Y\subseteq X\}
\]
eine Menge.
\end{axiom}
Wir nennen $\pot X$ die \Def{Potenzmenge} von $X$ und das Axiom, welches uns sichert, dass $\pot X$ tatsächlich eine Menge bildet das \Def{Potenzmengenaxiom}. Die Potenzmenge von $X$ enthält genau die Teilmengen von $X$.

\begin{axiom}
Sind $x$ und $y$ Mengen, so ist auch
\[
\{x,y\}=\{z\mid z=x\text{ oder }z=y\}\]
eine Menge.
\end{axiom}
Die Menge $\{x,y\}$ enthält genau die Mengen $x$ und $y$. Obiges Axiom heißt das \Def{Paarmengenaxiom}. Ist $x$ eine Menge, so schreiben wir für $\{x,x\}$ auch $\{x\}$ und nennen $\{x\}$ die \Def{einpunktige Menge}.

\begin{axiom}
Ist $\mathcal{X}$ eine Menge, so ist auch
\[
\bigcup\mathcal{X}=\left\{x\strich x\in X\text{ für ein gewisses }X\in\mathcal{X}\right\}
\]
eine Menge.
\end{axiom}
Wir nennen diese Axiom das \Def{Vereinigungsaxiom} und $\bigcup\mathcal{X}$ die \Def{Vereinigung} von $\mathcal{X}$. Sie enthält genau die Elemente von Elementen von $\mathcal{X}$. Sind insbesondere $X$ und $Y$ Mengen, so schreiben wir $X\cup Y$ für $\bigcup\{X,Y\}$. Diese Menge enthält alle Mengen, die entweder ein Element von $X$ oder ein Element von $Y$ sind.

Es seien $x$ und $y$ Mengen. Wir schreiben an dieser Stelle $(x,y)$ für $\{\{x\},\{x,y\}\}$. Dann gilt:

\begin{lemma}
Sind $X$ und $Y$ Mengen, so ist auch
\[
\ll X,Y\rr=\left\{z\strich z=(x,y)\text{ für gewisse }x\in X,y\in Y\right\}
\]
eine Menge.
\end{lemma}
\begin{proof}
Es gilt $\ll X,Y\rr\subseteq\pot\pot(X\cup Y)$. Die Behauptung folgt nun aus dem Teilmengenaxiom.
\end{proof}

\begin{lemma}
Für Mengen $x_1,x_2,y_1,y_2$ gilt $(x_1,x_2)=(y_1,y_2)$ genau dann, wenn $x_1=x_2$ und $y_1=y_2$.
\end{lemma}
\begin{proof}
Dass aus $x_1=x_2$ und $y_1=y_2$ folgt, dass $(x_1,x_2)=(y_1,y_2)$, ist klar.

Gelte daher $(x_1,x_2)=(y_1,y_2)$. Wir unterscheiden zwei Fälle. Nehmen wir an, dass $x_1=x_2$, dann gilt $\{\{x_1\}\}=(x_1,x_2)=(y_1,y_2)=\{\{y_1\},\{y_1,y_2\}\}$. Es gilt daher $\{x_1\}=\{y_1\}=\{y_1,y_2\}$. Aus der ersten Gleichheit folgt $x_1=y_2$. Aus der zweiten Gleichheit folgt dann $y_1=y_2$ und insgesamt, dass $x_1=x_2=y_1=y_2$.

Sind hingegen $x_1$ und $x_2$ verschieden, so muss auch $y_1$ von $y_2$ verschieden sein, da sonst obiges Argument analog greift. Insbesondere sind dann $\{x_1\}$ und $\{x_1,x_2\}$, sowie $\{y_1\}$ und $\{y_1,y_2\}$ verschieden. Aus
\begin{align}
\{\{x_1\},\{x_1,x_2\}\}=\{\{y_1\},\{y_1,y_2\}\}
\end{align}
folgt dann, dass $\{x_1\}=\{y_1\}$ oder $\{x_1\}=\{y_1,y_2\}$. Aus der zweiten Gleichung würde aber $y_1=x_1=y_2$ folgen, also muss die ersten Gleichung richtig sein, aus welcher $x_1=y_1$ folgt. Aus (1) folgt dann, dass $\{x_1,x_2\}=\{y_1,y_2\}$. Da $x_2$ von $x_1=y_1$ verschieden ist, aber $x_2\in\{y_1,y_2\}$, muss auch $x_2=y_2$ gelten.
\end{proof}

\section{Relationen und Abbildungen}

Es seien $X$ und $Y$ zwei Mengen und $\mathcal{R}$ eine Teilmenge von $\ll X,Y\rr$. Dann nennen wir $R=((X,Y),\mathcal{R})$ eine \Def{Relation}. Offenbar ist $R$ eindeutig durch die Mengen $X,Y$ und die Teilmenge $\mathcal{R}$ festgelegt. Anstatt $(x,y)\in\mathcal{R}$ schreiben wir auch $xRy$. Um die Relation $R$ festzulegen, schreiben wir
\[
X\xlongrightarrow{R}Y\,,\qquad xRy\iff(x,y)\in\mathcal{R}\,.
\]

Gilt $X\xlongrightarrow{R}Y$, so bezeichnen wir $X$ mit $\dom R$, $Y$ mit $\cod R$ und nennen diese Mengen die \Def{Domäne} bzw. \Def{Kodomäne} von $R$.

Haben wir zwei Relationen $X\xlongrightarrow{R}Y\xlongrightarrow{S}Z$ gegeben, so definieren wir die \Def{Komposition} durch
\[
X\xlongrightarrow{S\circ R}Z\,,\qquad x(S\circ R)z\iff xRy\text{ und }ySz\text{ für ein }y\in Y\,.
\]
\begin{task}
Sind drei Relationen $W\xlongrightarrow{R}X\xlongrightarrow{S}Y\xlongrightarrow{T}Z$ gegeben, so gilt
\[(T\circ S)\circ R=T\circ(S\circ R)\,.\]
\end{task}
Wir nennen eine Relation $X\xlongrightarrow{R}Y$ \Def{linkstotal}, wenn zu jedem $x\in X$ ein $y\in Y$ existiert mit $xRy$. Sie heißt \Def{rechtstotal}, wenn zu jedem $y\in Y$ ein $x\in X$ existiert mit $xRy$.
\begin{task}
Die Komposition linkstotaler Relationen ist linkstotal. Die Komposition rechtstotaler Relationen ist rechtstotal.
\end{task}
$X\xlongrightarrow{R}Y$ heißt \Def{linkseindeutig}, wenn zu jedem $y\in Y$ höchstens ein $x\in X$ existiert mit $xRy$. Sie heißt \Def{rechsteindeutig}, wenn zu jedem $x\in X$ höchstens ein $y\in Y$ existiert mit $xRy$.
\begin{task}
Die Komposition linkseindeutiger Relationen ist linkseindeutig. Die Komposition rechtseindeutiger Relationen ist rechtseindeutig.
\end{task}
Eine Relation $X\xlongrightarrow{f}Y$ heißt \Def{Abbildung}, wenn sie linkstotal und rechtseindeutig ist; das heißt, wenn zu jedem $x\in X$ genau ein $y\in Y$ existiert mit $xfy$. Dieses eindeutig bestimmte Element aus $Y$ bezeichnen wir mit $f(x)$ und nennen es das \Def{Bild} von $x$ unter $f$. Statt $f(x)$ schreiben wir auch $fx$ und vermeiden unnötige Klammern wann immer möglich. Eine Abbildung ist vollständig charakterisiert durch Angabe von Domäne, Kodomäne und für jedes Element aus der Domäne dessen Bild. Wir schreiben hierfür
\[
X\xlongrightarrow{f}Y\,,\qquad x\longmapsto fx
\]
oder auch
\[
X\xlongrightarrow{x\longmapsto fx}Y\,.
\]
Nach obigen Aufgaben ist die Komposition von Abbildungen $X\xlongrightarrow{f}Y\xlongrightarrow{g}Z$ wieder eine Abbildung. Sie ist offenbar gegeben durch
\[
X\xlongrightarrow{g\circ f}Z\,,\qquad x\longmapsto gfx\,.
\]

Eine Abbildung $X\xlongrightarrow{f}Y$ heißt \Def{injektiv}, wenn sie linkseindeutig ist. Das heißt, dass aus $fx_1=fx_2$ stets $x_1=x_2$ folgt. Sie heißt \Def{surjektiv}, wenn sie rechtstotal ist, das heißt, dass es zu jedem $y\in Y$ ein $x\in X$ gibt mit $fx=y$. Ein solches Element heißt \Def{Urbild} von $y$. Die Abbildung $f$ ist also genau dann injektiv, wenn jedes $y\in Y$ höchstens ein Urbild hat und genau dann surjektiv wenn jedes $y\in Y$ mindestens ein Urbild hat. $f$ heißt \Def{bijektiv}, wenn $f$ injektiv und surjektiv ist, wenn also jedes $y\in Y$ genau ein Urbild hat.

Es sei weiterhin $X\xlongrightarrow{f}Y$ eine Abbildung. Ist $A$ eine Teilmenge von $X$, so schreiben wir 
\[fA=\left\{y\in Y\strich y=fx\text{ für ein gewisses }x\in X\right\}\,.\]
$fA$ ist also die Menge aller Bilder von Elementen aus $A$. Wir nennen $fA$ auch das \Def{Bild} von $A$ unter $f$.

Ist $B$ eine Teilmenge von $Y$, so heißt
\[
f^{-1}B=\left\{x\in X\mid fx\in B\right\}
\]
das \Def{Urbild} von $B$ unter $f$.

\begin{example}
Ist $X$ Teilmenge einer Menge $Y$, so haben wir die \Def{Inklustionsabbildung}
\[
X\xlongrightarrow{\iota}Y\,,\qquad x\longmapsto x\,,
\]
welche offensichtlich injektiv ist. Falls $X=Y$ gilt, ist $\iota$ sogar bijektiv und wir schreiben
\[
X\xlongrightarrow{\id_X}X\,,\qquad x\longmapsto x\,.
\]
Für jede Menge $X$ nennen wir $\id_X$ die \emph{Identität} auf $X$. Sind $R,Y$ beliebige Relationen
\[
W\xlongrightarrow{R}X\xlongrightarrow{\id_X}X\xlongrightarrow{S}Y\,,
\]
so gilt offenbar $\id_X\circ R=R$ und $S\circ\id_X=\id_X$.
\end{example}

Sind $X,Y$ Klassen und nicht notwendigerweise Mengen, so nennen wir eine Teilmenge $R\subseteq\ll X,Y\rr$ eine \Def{Relation} zwischen $X$ und $Y$, ohne Domäne und Kodomäne zum Teil der Definition zu machen. Die Begriffe \Def{linkstotal}, \Def{rechtstotal}, \Def{linkseindeutig}, \Def{rechtseindeutig} und \Def{Abbildung} verwenden wir analog.

\begin{axiom}
Ist $U\xlongrightarrow{f}U$ eine Abbildung aus der Allklasse in sich selbst und ist $X$ eine Menge, so ist auch $fX$ eine Menge.
\end{axiom}

Dieses \Def{Ersetzungsaxiom} ist äquivalent zur folgenden stärkeren Formulierung:

\begin{task}
Es sei $V\xlongrightarrow{f}W$ eine Abbildung von Klassen. Für jede Menge $X\subseteq V$ ist dann auch $fX$ eine Menge. 
\end{task}

\section{Mengentheoretische Konstruktionen}

Eine Abbildung $I\xlongrightarrow{x}X$ von Mengen nennen wir manchmal eine \Def{Familie}. In diesem Fall schreiben wir häufig $x_i$ für das Bild von $i$ unter $x$ und $(x_i)_{i\in I}$ für die Familie $x$. Die Kodomäne nennen wir in diesem Fall meist nicht. Die Menge $I$ nennen wir in dieser Situation die \Def{Indexmenge}. Ein Element aus der Indexmenge heißt ein \Def{Index}. Für $i\in I$ nennen wir $x_i$ die \Def{Komponente} von $(x_i)_{i\in I}$ an der Stelle $i$. Mit $X^I$ bezeichnen wir die Menge aller durch $I$ indizierten Familien mit Komponenten in $X$. 

Ist $(X_i)_{i\in I}$ eine Familie von Mengen, so bezeichnen wir mit
\[
\bigcup_{i\in I}X_i=\left\{x\mid x\in X_i\text{ für ein }i\in I\right\}
\]
die \Def{Vereinigung} der Familie. Schreiben wir die Familie als Abbildung $I\xlongrightarrow{X}Y$, so gilt offenbar $\bigcup_{i\in I}X_i=\bigcup XI$ und die Vereinigung der Familie bildet nach dem Vereinigungsaxiom sogar eine Menge.

Mit
\[
\bigcap_{i\in I}X_i=\left\{x\mid x\in X_i\text{ für alle } i\in I\right\}
\]
bezeichnen wir den \Def{Durchschnitt} der Familie. Falls $I$ leer ist, ist der Durchschnitt die Allklasse. Andernfalls ist der Durchschnitt offenbar in der Vereinigung enthalten und bildet nach dem Teilmengenaxiom eine Menge.

Die Menge
\[
\prod_{i\in I}X_i=\left\{(x_i)_{i\in I}\in\left(\bigcup_{i\in I}X_i\right)^I\strich x_i\in X_i\text{ für alle} i\in I\right\}
\]
heißt das \Def{kartesische Produkt} der Familie.

Für jedes $i\in I$ haben wir eine Abbildung
\[
\prod_{i\in I}\xlongrightarrow{\pi_i}X_i\,,\qquad(x_i)_{i\in I}\longmapsto x_i\,,
\]
welche wir die \Def{Projektion} auf die $i$-te Komponente nennen.
\begin{task}
Es sei $(X_i)_{i\in I}$ eine Familie von Mengen. Ist $W$ eine Mengen und $(W\xlongrightarrow{f_i}X_i)_{i\in I}$ eine Familie von Abbildungen, so existiert genau eine Abbildung
\[W\xlongrightarrow{f}\prod_{i\in I}X_i\]
mit $f_i=\pi_i\circ f$ für alle $i\in I$.
\end{task}

Es seien zwei Mengen $X_1,X_2$ gegeben. In diesem Fall bezeichnen wir das kartesische Produkt
\[
\prod_{i\in \{1,2\}}X_i
\]
mit $X_1\times X_2$. Für ein Element $(x_i)_{i\in\{1,2\}}\in X_1\times X_2$ schreiben wir $\l x_1,x_2\r$ und nennen es ein \Def{Paar}. Haben wir noch eine dritte Menge $X_3$, so schreiben wir $X_1\times X_2\times X_3$ für das Produkt $\prod_{i\in\{1,2,3\}}X_i$ und schreiben $\l x_1,x_2,x_3\r$ für ein Element $(x_i)_{i\in\{1,2,3\}}$ des Produktes. Ein solches Element heißt \Def{Tripel}.

\begin{example}
Es seien $X_1\xlongrightarrow{g_1}W_1$ und $X_2\xlongrightarrow{g_2}W_2$ zwei Abbildungen. Wir setzen $W=X_1\times W_2$, $f_1=g_1\circ\pi_1$, $f_2=g_2\circ\pi_2$. Nach obiger Aufgabe existiert genau eine Abbildung $W_1\times W_2\xlongrightarrow{f}X_1\times X_2$, sodass $g_1\circ\pi_1=\pi_1\circ f$ und $g_2\circ\pi_2=\pi_2\circ f$ (wobei wir hier natürlich $\pi_i$ für die Projektionen sowohl von $X_1\times X_2$ als auch für die von $W_1\times W_2$ schreiben). Die eindeutig bestimmte Abbildung $f$ bezeichnen wir mit $g_1\times g_2$. Sie ist gegeben durch
\[
W_1\times W_2\xlongrightarrow{g_1\times g_2}X_1\times X_2\,,\qquad \l w_1,w_2\r\longmapsto\l g_1w_1,g_2w_2\r\,.
\]
\end{example}
Mithilfe des kartesischen Produktes können wir das \Def{Auswahlaxiom} leicht formulieren
\begin{axiom}
Das kartesische Produkt einer Familie von nichtleeren Mengen ist nichtleer.
\end{axiom}

\section{Universen}

\begin{axiom}
Es existiert eine Menge $\IN$ zusammen mit zwei Abbildungen $\{\emptyset\}\xlongrightarrow{z}\IN\xlongrightarrow{s}\IN$, sodass für jede Menge $A$ zusammen mit zwei Abbildungen $\{\emptyset\}\xlongrightarrow{z'}A\xlongrightarrow{s'}A$ genau eine Abbildung $\IN\xlongrightarrow{r}A$ existiert, sodass $r\circ z=z'$ und $r\circ s=s'\circ r$.
\end{axiom}
Die Auswirkungen dieses \Def{Unendlicheitsaxioms} werden wir später studieren.

Eine Menge $\U$ heißt ein \Def{Universum}, wenn gilt:
\begin{itemize}
\item Aus $X\in\U$ folgt $\pot X\in\U$.
\item Aus $X\in\U$ folgt $X\subseteq U$.
\item Gilt $I\in\U$ und sit $(X_i)_{i\in I}$ eine Familie von Mengen aus $\U$, so gilt $\prod_{i\in I}X_i\in\U$.
\item Gilt $I\in\U$ und sit $(X_i)_{i\in I}$ eine Familie von Mengen aus $\U$, so gilt $\bigcup{i\in I}X_i\in\U$.
\item Ist $X\in\U$, $Y\subseteq\U$ und $X\xlongrightarrow{f}Y$ eine Abbildung, so gilt auch $fX\in\U$.
\item $\IN\in\U$.
\end{itemize}

Man beachte, dass die Allklasse $U$ allen Bedingungen genügt, außer dass sie selbst keine Menge ist.

Ist $\U$ ein Universum, so kann man sich überlegen, dass ferner gilt:

\begin{itemize}
\item $\emptyset\in\U$
\item Aus $X\in\U$ und $x\in X$ folgt $x\in\U$.
\item Aus $x,y\in\U$ folgt $\{x,y\}\in\U$.
\item Aus $I,X\in\U$ folgt $X^I\in\U$.
\item Sind $X,Y\in\U$ und ist $X\xlongrightarrow{R}Y$ eine Relation, so gilt $R\in\U$.
\item Ist $Y\in\U$ und gilt $X\subseteq Y$, so folgt $X\in\U$.
\end{itemize}

In anderen Worten können wir alle Konstruktionen, die wir in $U$ eingeführt haben, auch in jedem Universum $\U$ durchführen. Zusammen mit dem nachfolgenden \Def{Universenaxiom} sichert dies, dass man Mathematik sorgenfrei innerhalb eines Universums betreiben kann.

\begin{axiom}
Jede Menge ist Element eines Universums.
\end{axiom}

Eine Menge heißt \Def{$\U$-klein}, wenn sie Element von $\U$ ist. Ist klar oder nicht wichtig, welches Universum wir betrachten, nennen wir die Menge auch einfach \Def{klein}. Wann immer wir von kleinen Mengen sprechen, setzen wir ein Universum voraus, auf das wir uns beziehen.
